\documentclass[11pt,a4paper]{resume}
\usepackage{zh_CN-Adobefonts_external} % Simplified Chinese Support using external fonts (./fonts/zh_CN-Adobe/)
%\usepackage{zh_CN-Adobefonts_internal} % Simplified Chinese Support using system fonts
\usepackage{linespacing_fix} % disable extra space before next section
\usepackage{cite}
\usepackage{color}
\usepackage{tikz}  % 引入 TikZ 用于绝对定位绘图
\usepackage{fancyhdr}
\usepackage{changepage}
\geometry{left=0.7cm, right=0.7cm, top=1.3cm, bottom=0.7cm, headheight=0cm, headsep=0cm}
\pagestyle{fancy}

% 定义颜色
\definecolor{blueheader}{RGB}{0, 120, 190}
\definecolor{lightblue}{RGB}{230, 248, 255}

\fancyhf{} 
\renewcommand{\headrulewidth}{0pt}
\fancyhead[C]{%
  \begin{tikzpicture}[remember picture, overlay]
    \node[fill=blueheader, minimum width=\paperwidth, minimum height=0.8cm, anchor=north] at (current page.north) {};
  \end{tikzpicture}%
}

% 第一页特殊设置
\fancypagestyle{firstpage}{
  \fancyhf{}
  \renewcommand{\headrulewidth}{0pt}
  \fancyhead[C]{%
    \begin{tikzpicture}[remember picture, overlay]
      \node[fill=blueheader, minimum width=\paperwidth, minimum height=0.8cm, anchor=north] at (current page.north) {};
    \end{tikzpicture}%
  }
}

\newcommand{\sectionheader}[1]{%
  % \noindent
  \vspace{0.2cm}
  {
    \setlength{\fboxsep}{0pt}% 去除外框内边距
    \colorbox{lightblue}{%
      \parbox{\textwidth}{%
        \textcolor{blueheader}{\rule{8pt}{24pt}}% 宽度10pt,高度24pt
        \hspace{0.5cm}%
        \raisebox{6pt}{\color{blueheader}\Large\textbf{#1}}%
      }%
    }%
  }%
  \par\vspace{0.1cm}%
}


\begin{document}
\thispagestyle{firstpage}
\pagenumbering{gobble} % suppress displaying page number


% 用户只需要替换下面这些占位符即可:
%   [YOUR_NAME] [YOUR_PHONE] [YOUR_EMAIL] [YOUR_AGE] [YOUR_TITLE] [AVATAR_PATH]
% 若不想显示头像:删掉第三列 \includegraphics 那一行,或把它注释掉。

\renewcommand{\arraystretch}{1.4}
\begin{tabular}{@{} p{0cm} p{16.6cm} r @{}} 
    & {\Huge \heiti{\textbf{[Name]}}}
    &
    % 头像.png替代1.png即可
    \raisebox{-1.55cm}[0pt][0pt]{\includegraphics[height=2.6cm]{1.png}} \\
    \\[-0.2cm]
    & \textbf{[Phone]} $|$ \textbf{[Email]} $|$  \\
    & [兴趣爱好] & \\
\end{tabular}

% Education
\sectionheader{教育背景}
\begin{adjustwidth}{1em}{0em}
    \textbf{西南财经大学}(211/双一流),计算机科学与技术,\textit{在读硕士研究生} \hfill (2025.9 -- 2028.6) \\
    \textbf{西南财经大学}(211/双一流),计算机科学与技术,\textit{工学学士} \hfill (2021.9 -- 2025.6) \\
    \hspace*{1em}\textbf{排名 [X]/[N](前 [X]\%)},\textbf{[奖学金/各类在校荣誉称号]}
\end{adjustwidth}


\sectionheader{科研经历}
\begin{adjustwidth}{1em}{0em}
    \textbf{论文标题}:\textit{[Paper\_Title]} \\
    \textbf{作者角色}:[共同第一作者/第一作者/学生作者] \hfill(\textbf{Venue/Journal})\\
    \textbf{内容概述}:[用 2--3 句描述研究目标、核心方法、结论。]\\
    \textbf{主要贡献}:[用 2--3 句描述你做了什么:实现/实验/消融/分析/部署。]
\end{adjustwidth}


\sectionheader{项目经历/实习经历}
\begin{adjustwidth}{1em}{0em}
    \datedsubsection{\textbf{[Project\_Title\_1]}}{[YYYY.MM] -- [YYYY.MM]}
    \textbf{内容概述}:[说明问题背景 + 目标。]\\
    \textbf{主要工作产出}:[在这里主要写你做了什么,分点式写法也可以。]
    \datedsubsection{\textbf{[Project\_Title\_2]}}{[YYYY.MM] -- [YYYY.MM]}
    \textbf{内容概述}:[--------------------------------------------------]\\
    \textbf{主要工作产出}:[---------------------------------------------------]
\end{adjustwidth}

\sectionheader{比赛经历}
\begin{adjustwidth}{1em}{0em}
    \datedsubsection{\textbf{[Competition\_1]}}{[YYYY.MM] -- [YYYY.MM]}
    \textbf{内容概述}:[任务是什么,最终名次(如:第 X 名 / Top X\%)。]\\
    \textbf{主要产出}:[方法/结果]
    \datedsubsection{\textbf{[Cpmpetition\_2]}}{[YYYY.MM] -- [YYYY.MM]}
    \textbf{内容概述}:     [---------------------------------------------------]\\
    \textbf{主要产出}:  [---------------------------------------------------]
\end{adjustwidth}

% =========================
% Skills
% =========================
\sectionheader{技能特长}
\begin{adjustwidth}{1em}{0em}
    \begin{itemize}[parsep=0.2ex]
        \item \textbf{英语}:CET-4/CET-6。
        \item \textbf{技术}:Python / PyTorch;Pandas 数据处理、特征工程与可视化;熟悉 [NLP/多模态/分布式训练/部署] 等方向的常用工具。
    \end{itemize}
\end{adjustwidth}


\end{document}
